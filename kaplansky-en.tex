\documentclass[12pt]{scrartcl}
\usepackage[utf8]{inputenc}
\usepackage[english]{babel}
\usepackage{amsthm,amssymb,mathtools}
\usepackage[protrusion=true,expansion=true]{microtype}

\usepackage[T1]{fontenc}
\usepackage{libertine}

\theoremstyle{definition}
\newtheorem*{defn}{Definition}

\theoremstyle{plain}
\newtheorem*{prop}{Proposition}
\newtheorem*{cor}{Corollary}
\newtheorem*{lemma}{Lemma}

\theoremstyle{remark}
\newtheorem*{rem}{Remark}

\newcommand{\defeq}{\vcentcolon=}
\newcommand{\defequiv}{\vcentcolon\equiv}
\newcommand{\seq}[1]{\mathrel{\vdash\!\!\!_{#1}}}

\newcommand{\aaa}{\mathfrak{a}}
\newcommand{\Hom}{\mathrm{Hom}}

\pagestyle{empty}

\begin{document}

\enlargethispage{4em}

\begin{center}\huge\textbf{\textsf{Vector bundles on affine schemes}}\end{center}
\bigskip

\begin{lemma}Let~$A$ be a local ring. Let~$\aaa$ be a finitely generated
idempotent ideal in $A$. Then~$\aaa = (0)$ or~$\aaa = (1)$.
\end{lemma}

\begin{proof}Consider~$\aaa$ as a finitely generated~$A$-module. Then, by
Nakayama's lemma, there exists an element~$x \in A$ such that~$x \equiv 1$
modulo~$\aaa$ and~$x \aaa = 0$. Since~$A$ is a local ring, $x$ is invertible
or~$1-x$ is invertible. In the first case it follows that~$\aaa = (0)$, in the
second that~$\aaa = (1)$.
\end{proof}

\begin{lemma}Let~$A$ be a local ring. Let~$P$ be an idempotent matrix over~$A$.
Then~$P$ is equivalent to a diagonal matrix with entries~$1$ and~$0$.
\end{lemma}

\begin{proof}Since~$P$ is idempotent, so are its ideals~$(\Lambda^i P)$
of~$i$-minors:
\[ (\Lambda^i P) = (\Lambda^i (P \circ P)) =
(\Lambda^i P \circ \Lambda^i P) \subseteq (\Lambda^i P) \cdot (\Lambda^i P)
\subseteq (\Lambda^i P). \]
By the previous lemma, they are therefore each equal to~$(0)$ or~$(1)$. Since
they form a descending chain, there exists a stage~$r$ such that~$(\Lambda^r P)
= (1)$ and~$(\Lambda^{r+1} P) = (0)$. Therefore all~$(r+1)$-minors of~$P$ are
zero, and -- since~$A$ is a local ring -- there exists at least one
invertible~$r$-minor. Thus~$P$ can be made into a diagonal matrix of the
desired form by applying row and column transformations.
\end{proof}

\begin{rem}We can even show that~$P$ is \emph{similar} to a diagonal matrix
with entries~$1$ and~$0$: By the lemma, image and kernel of~$P$ are finite
free. Combining bases of these subspaces, we obtain a basis of the full space;
expressing~$P$ with respect to this basis, we obtain a diagonal matrix of the
desired form.\end{rem}

\begin{prop}An~$A$-module~$M$ is finitely generated and projective if and only
if there exists a partition~$1 = \sum_i f_i \in A$ such that the localized
modules~$M[f_i^{-1}]$ are each finite free over~$A[f_i^{-1}]$.\end{prop}

\begin{proof}Let~$M$ be a finitely generated and projective~$A$-module. Then
there exists a linear surjection~$p : A^n \to M$ with a section~$s : M \to
A^n$. The composition~$P \defeq s \circ p$ is idempotent and~$M$ is isomorphic
to~$A^n/\operatorname{ker}(P)$. Interpreting the previous lemma in the little
Zariski topos of~$\operatorname{Spec} A$, we see that there exists a partition
of unity such that~$P$ is, over each of the localized rings, equivalent to a
diagonal matrix with entries~$1$ and~$0$. Since localization is exact, the
module~$A^n/\operatorname{ker}(P)$ is therefore finite locally free.

Conversely, let~$M$ be a finite locally free~$A$-module. Then~$M$ is locally
finitely generated and therefore also globally finitely generated. Fix a linear
surjection~$A^n \to M$. Its kernel is finitely generated, since localization is
exact and~$M$ is locally finitely presented. Thus the kernel is also globally
finitely generated. This shows that~$M$ is finitely presented.

To verify that~$M$ is projective, consider an arbitrary linear surjection~$X
\to Y$. We have to show that the postcomposition map~$\Hom_A(M,X) \to
\Hom_A(M,Y)$ is too surjective. Since~$M$ is locally projective and
since~$\Hom_A(M,\cdot)$ commutes with localization (because~$M$ is finitely
presented), this map is locally surjective and therefore surjective.
\end{proof}

\begin{cor}Let~$M$ be an~$A$-module. The induced quasicoherent sheaf of
modules~$M^\sim$ on~$\operatorname{Spec} A$ is a vector bundle if and only
if~$M$ is finitely generated and projective.\end{cor}

\end{document}
