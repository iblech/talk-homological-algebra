\documentclass[12pt]{scrartcl}
\usepackage[utf8]{inputenc}
\usepackage[ngerman]{babel}
\usepackage{etex}
\usepackage{amsmath,amsthm,amssymb,color}
\usepackage{mathtools}
\usepackage[protrusion=true,expansion=true]{microtype}
\usepackage{hyperref}
\usepackage[all]{xy}

\usepackage[T1]{fontenc}
\usepackage{libertine}
\linespread{1.05}

\theoremstyle{definition}
\newtheorem*{defn}{Definition}

\theoremstyle{plain}
\newtheorem*{prop}{Proposition}
\newtheorem*{cor}{Korollar}
\newtheorem*{lemma}{Lemma}

\theoremstyle{remark}
\newtheorem*{rem}{Bemerkung}

\newcommand{\defeq}{\vcentcolon=}
\newcommand{\defequiv}{\vcentcolon\equiv}
\newcommand{\seq}[1]{\mathrel{\vdash\!\!\!_{#1}}}

\newcommand{\aaa}{\mathfrak{a}}
\newcommand{\Hom}{\mathrm{Hom}}

\pagestyle{empty}

\begin{document}

\enlargethispage{4em}

\begin{center}\huge\textbf{\textsf{Vektorbündel auf affinen Schemata}}\end{center}
\bigskip

\begin{lemma}Sei~$A$ ein lokaler Ring. Sei~$\aaa$ ein endlich erzeugtes
idempotentes Ideal in~$A$. Dann ist~$\aaa = (0)$ oder~$\aaa = (1)$.\end{lemma}

\begin{proof}Wir können~$\aaa$ als endlich erzeugten~$A$-Modul ansehen. Dann
gibt es nach dem Lemma von Nakayama ein Element~$x \in A$ mit~$x \equiv 1$
modulo~$\aaa$ und~$x \aaa = 0$. Da~$A$ lokal ist, ist~$x$ invertierbar
oder~$1-x$ invertierbar. Im ersten Fall ist~$\aaa = (0)$, im zweiten~$\aaa =
(1)$.\end{proof}

\begin{lemma}Sei~$A$ ein lokaler Ring. Sei~$P$ eine idempotente Matrix
über~$A$. Dann ist~$P$ äquivalent zu einer Diagonalmatrix mit Einträgen~$1$
und~$0$.\end{lemma}

\begin{proof}Da~$P$ idempotent ist, sind auch die Ideale~$(\Lambda^i P)$
der~$i$-Minoren idempotent. Nach dem vorherigen Lemma sind sie daher jeweils
Null oder Eins. Da sie eine absteigende Kette bilden, gibt es somit eine
Zahl~$r$, sodass~$(\Lambda^r P) = (1)$ und~$(\Lambda^{r+1} P) = (0)$. Also sind
alle~$(r+1)$-Minoren von~$P$ Null, und -- da~$A$ lokal ist -- gibt es
mindestens einen invertierbaren~$r$-Minor. Somit lässt sich~$P$ durch Zeilen-
und Spaltentransformationen auf eine Diagonalmatrix der gewünschten Art
bringen. \end{proof}

\begin{rem}Man kann sogar zeigen, dass~$P$ \emph{ähnlich} zu einer
Diagonalmatrix dieser Art ist. Denn mit dem Lemma ist klar, dass Bild und Kern
von~$P$ frei von endlichem Rang sind. Kombiniert man Basen von Bild und Kern,
so erhält man eine Basis des vollen Spaltenraums, bezüglich der die
entsprechend transformierte Matrix eine Diagonalmatrix der gewünschten Art
ist.\end{rem}

\begin{prop}Sei~$M$ ein Modul über einem Ring~$A$. Dann ist~$M$ genau dann
endlich erzeugt und projektiv, wenn es eine Zerlegung~$1 = \sum_i f_i \in A$
gibt, sodass die lokalisierten Moduln~$M[f_i^{-1}]$ jeweils freie Moduln
endlichen Rangs sind.\end{prop}

\begin{proof}Sei~$M$ endlich erzeugt und projektiv. Dann gibt es eine lineare
Surjektion~$p : A^n \to M$ mit Schnitt~$s : M \to A^n$. Die Komposition~$P
\defeq sp$ ist idempotent, und~$M$ ist isomorph zu~$A^n/\operatorname{ker}(P)$.
Interpretiert man das vorherige Lemma im kleinen Zariski-Topos
über~$\operatorname{Spec} A$, sieht man, dass eine Zerlegung der Eins
existiert, sodass~$P$ über den lokalisierten Ringen jeweils äquivalent zu einer
Diagonalmatrix mit Einträgen~$1$ und~$0$ ist. Da Lokalisieren exakt ist, ist
daher~$A^n/\operatorname{ker}(P)$ lokal frei von endlichem Rang.

Sei umgekehrt ein Modul~$M$ gegeben, der lokal frei von endlichem Rang ist.
Dann ist~$M$ auch global endlich erzeugt. Sei~$A^n \to M$ eine lineare
Surjektion. Ihr Kern ist lokal endlich erzeugt, da Lokalisieren exakt ist
und~$M$ lokal endlich präsentiert ist. Damit ist ihr Kern auch global endlich
erzeugt; folglich ist~$M$ von endlicher Präsentation. Um nun die Projektivität
von~$M$ nachzuweisen, sei eine lineare Surjektion~$X \to Y$ gegeben. Es ist zu
zeigen, dass die induzierte Abbildung~$\Hom_A(M,X) \to \Hom_A(M,Y)$ ebenfalls
surjektiv ist. Da~$M$ lokal projektiv ist, und~$\Hom_A(M,\cdot)$ aufgrund der
endlichen Präsentation von~$M$ mit Lokalisieren vertauscht, ist diese Abbildung
lokal surjektiv und somit surjektiv.
\end{proof}

\begin{cor}Sei~$M$ ein Modul über einem Ring~$A$. Dann ist die induzierte
quasikohärente Modulgarbe~$M^\sim$ genau dann ein Vektorbündel, wenn~$M$
endlich erzeugt und projektiv ist.\end{cor}

%\begin{proof}Wegen der Äquivalenz zwischen quasikohärenten Modulgarben auf
%affinen Schemata und gewöhnlichen Moduln auf Ringen ist die Modulgarbe~$M^\sim$
%genau dann ein Vektorbündel, also lokal frei von endlichem Rang, wenn~$M$ lokal
%frei von endlichem Rang ist.\end{proof}

\end{document}
