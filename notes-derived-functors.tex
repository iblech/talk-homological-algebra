\documentclass{amsart}
\usepackage[utf8]{inputenc}
\usepackage[english]{babel}
\usepackage{etex}
\usepackage{amsmath,amsthm,amssymb,stmaryrd,color,graphicx,multirow}
\usepackage{mathtools}
\usepackage{xspace}
\usepackage[protrusion=true,expansion=true]{microtype}
\usepackage{hyperref}
\usepackage[all]{xy}

\usepackage{mathpazo}
\linespread{1.05}

\theoremstyle{definition}
\newtheorem{defn}{Definition}[section]
\newtheorem{ex}[defn]{Example}

\theoremstyle{plain}

\newtheorem{prop}[defn]{Proposition}
\newtheorem{cor}[defn]{Corollary}
\newtheorem{lemma}[defn]{Lemma}
\newtheorem{thm}[defn]{Theorem}

\theoremstyle{remark}
\newtheorem{rem}[defn]{Remark}

\newcommand{\ZZ}{\mathbb{Z}}
\newcommand{\FF}{\mathbb{F}}
\renewcommand{\AA}{\mathbb{A}}
\newcommand{\A}{\mathcal{A}}
\newcommand{\B}{\mathcal{B}}
\newcommand{\C}{\mathcal{C}}
\newcommand{\D}{\mathcal{D}}
\newcommand{\E}{\mathcal{E}}
\newcommand{\F}{\mathcal{F}}
\newcommand{\G}{\mathcal{G}}
\renewcommand{\H}{\mathcal{H}}
\renewcommand{\O}{\mathcal{O}}
\newcommand{\K}{\mathcal{K}}
\newcommand{\N}{\mathcal{N}}
\newcommand{\M}{\mathcal{M}}
\renewcommand{\L}{\mathcal{L}}
\renewcommand{\P}{\mathcal{P}}
\newcommand{\R}{\mathcal{R}}
\newcommand{\I}{\mathcal{I}}
\renewcommand{\S}{\mathcal{S}}
\newcommand{\NN}{\mathbb{N}}
\newcommand{\RR}{\mathbb{R}}
\newcommand{\LL}{\mathbb{L}}
\newcommand{\QQ}{\mathbb{Q}}
\newcommand{\GG}{\mathbb{G}}
\newcommand{\aaa}{\mathfrak{a}}
\newcommand{\ppp}{\mathfrak{p}}
\newcommand{\mmm}{\mathfrak{m}}
\newcommand{\nnn}{\mathfrak{n}}
\newcommand{\Ext}{\mathrm{Ext}}
\newcommand{\Tor}{\mathrm{Tor}}
\newcommand{\Hom}{\mathrm{Hom}}
\newcommand{\HOM}{\mathcal{H}\mathrm{om}}
\newcommand{\id}{\mathrm{id}}
\newcommand{\GL}{\mathrm{GL}}
\newcommand{\placeholder}{\underline{\ \ \ }}
\newcommand{\ul}[1]{\underline{#1}}
\newcommand{\Set}{\mathrm{Set}}
\newcommand{\Grp}{\mathrm{Grp}}
\newcommand{\Vect}{\mathrm{Vect}}
\newcommand{\Sh}{\mathrm{Sh}}
\newcommand{\PSh}{\mathrm{PSh}}
\newcommand{\Zar}{\mathrm{Zar}}
\newcommand{\Sch}{\mathrm{Sch}}
\newcommand{\Ab}{\mathrm{Ab}}
\newcommand{\AbSh}{\mathrm{AbSh}}
\newcommand{\Mod}{\mathrm{Mod}}
\newcommand{\Alg}{\mathrm{Alg}}
\newcommand{\Ring}{\mathrm{Ring}}
\newcommand{\LRL}{\mathrm{LRL}}
\newcommand{\pt}{\mathrm{pt}}
\newcommand{\tors}{\mathrm{tors}}
\DeclareMathOperator{\Spec}{Spec}
\newcommand{\QcohSpec}[2]{\mathrm{Spec}^{\mathrm{qcoh}}_{#1}{#2}}
\newcommand{\RelSpec}[2]{\mathrm{RelSpec}_{#1}{#2}}
\newcommand{\op}{\mathrm{op}}
\DeclareMathOperator{\colim}{colim}
\DeclareMathOperator{\rank}{rank}
\DeclareMathOperator{\Ann}{Ann}
\DeclareMathOperator{\Int}{int}
\DeclareMathOperator{\Clos}{cl}
\DeclareMathOperator{\Kernel}{ker}
\DeclareMathOperator{\Cokernel}{cok}
\DeclareMathOperator{\supp}{supp}
\newcommand{\Ass}{\mathrm{Ass}}
\newcommand{\Open}{\mathrm{Op}}
\newcommand{\?}{\,{:}\,}
\renewcommand{\_}{\mathpunct{.}\,}
\newcommand{\speak}[1]{\ulcorner\text{\textnormal{#1}}\urcorner}
\newcommand{\Ll}{\vcentcolon\Longleftrightarrow}
\newcommand{\notat}[1]{{!#1}}
\newcommand{\lra}{\longrightarrow}
\newcommand{\lhra}{\ensuremath{\lhook\joinrel\relbar\joinrel\rightarrow}}
\newcommand{\hra}{\hookrightarrow}
\newcommand{\brak}[1]{{\llbracket{#1}\rrbracket}}
\newcommand{\ie}{i.\,e.\@\xspace}
\newcommand{\eg}{e.\,g.\@\xspace}
\newcommand{\vs}{vs.\@\xspace}
\newcommand{\resp}{resp.\@\xspace}
\newcommand{\inv}{inv.\@}
\newcommand{\notnot}{\emph{not not}\xspace}
\newcommand{\dd}{{\vee\!\!\!\vee}}

\newcommand{\XXX}[1]{\textbf{XXX: #1}}

\newcommand{\defeq}{\vcentcolon=}
\newcommand{\defequiv}{\vcentcolon\equiv}
\newcommand{\seq}[1]{\mathrel{\vdash\!\!\!_{#1}}}

\newcommand{\stackhref}[1]{\href{http://stacks.math.columbia.edu/tag/#1}{#1}}

\title{A quickstart guide to derived functors}
\author{Ingo Blechschmidt}
%\email{iblech@web.de}

\begin{document}

\maketitle

\section{References}

\section{Abelian categories}

\begin{itemize}
\item Definition
\item Examples
\item Counter examples
\item Diagram chases
\item Uses: category of coefficient objects, source and target of functors to
derive
\item Exact functors: classical and finite limit definition, in footnote: what,
if one doesn't start with \emph{short} exact sequences?
\end{itemize}

\section{Injectives and projectives}

For the theory of derived functors, injective and projective objects are very
important. We'll discuss why below.

\begin{defn}An object~$I$ of an abelian category is \emph{injective} if and
only if, for any monomorphism~$X \hookrightarrow Y$ and any morphism~$X \to
I$, there exists a lift of that morphism to~$Y$.
\[ \xymatrix{
  0 \ar[r] & X \ar@{^{(}->}[r] \ar[d] & Y \ar@{-->}[ld] \\
  & I
} \]
Dually, an object~$P$ is \emph{projective} if and only if, for any
epimorphism~$X \twoheadrightarrow Y$ and any morphism~$P \to X$, there exists a
co-lift of that morphism to~$Y$.
\[ \xymatrix{
   X \ar@{->>}[r] & Y \ar[r] & 0 \\
  & P \ar[u] \ar@{-->}[lu]
} \]
\end{defn}

Note that no uniqueness of the lifts is required.

\begin{ex}In the category of vector spaces over some field, assuming the axiom
of choice, any object~$I$ is injective: Simply take a basis of~$X$, extend it to a
basis of~$Y$, and define~$Y \to I$ by zero on the extended zero. This gives
further indication that categories of vector spaces are not very interesting
from a homological point of view.\end{ex}

\begin{ex}\emph{Baer's criterion} states that in the category of abelian
groups, assuming the axiom of choice, a group~$G$ is injective if and only if
it is \emph{divisible}, \ie if for any~$x \in G$ and~$n \geq 1$ there exists an
element~$y \in G$ such that~$x = ny$.\footnote{The ``only if'' direction is
easy: Let~$x \in G$ and~$n \geq 1$. Then consider the injective map~$\ZZ
\xrightarrow{n} \ZZ$. The sought element~$y$ can be obtained as the image
of~$1$ under a lift of the map~$\ZZ \to G,\,1 \mapsto x$.} For example, the
groups~$\QQ$ and~$\QQ/\ZZ$ are injective.
\end{ex}

Injective objects are generally regarded as huge, unwieldy objects. They are
important for the theory, but no practical calculations are made with them.
Projective objects, on the other hand, are often much more accessible. This is
of course a purely social statement, since the injective objects in an abelian
category~$\A$ are precisely the projective objects in~$\A^\op$.

\begin{ex}In the category of modules over some ring, any finite free module~$P$ is
projective: Choose a basis of~$P$ and pick preimages under~$X
\twoheadrightarrow Y$ of the images of the basis vectors in~$Y$. Use these
preimages to define the co-lift~$P \to X$. More generally, assuming the axiom
of choice, any (not necessarily finite) free module is projective. The precise
characterization is that a module is projective if and only if it is a direct
summand of a free module.\footnote{The relationship to other properties
of~$R$-modules is as follows: \\
\[ \xymatrixcolsep{5pc}\xymatrix{
  \text{free} \ar@/^1pc/@{=>}[r] &
  \text{projective} \ar@/^1pc/@{=>}[r] \ar@/^1pc/@{==>}[l]^{\text{$R$ local
  or $R$ PID}} &
  \text{flat} \ar@/^1pc/@{=>}[r] \ar@/^1pc/@{==>}[l]^{\text{$R$ perfect}} &
  \text{torsionfree} \ar@/^1pc/@{==>}[l]^{\text{$R$ Dedekind domain}}
} \]}\end{ex}

\begin{ex}Assuming the axiom of choice, any vector space is free and thus
projective.\end{ex}

\begin{ex}An easy example of a projective module which is not free is
the~$\ZZ/(6)$-module~$\ZZ/(2)$. It is projective because there is the direct sum
decomposition $\ZZ/(6) \cong \ZZ/(2) \oplus \ZZ/(3)$. It is not free because
its number of elements is not a multiple of~$6$.\end{ex}

\begin{ex}Let~$\mmm = (3, 1 + \sqrt{-5})$ be the famous example of an ideal
of~$R = \ZZ[\sqrt{-5}]$ which is locally, but not globally principal.
Then~$\mmm$ is projective but not free.\end{ex}

\begin{ex}By the
\href{http://ncatlab.org/nlab/show/Serre-Swan+theorem}{\emph{Serre--Swan
theorem}}, vector bundles provide a systematic source of projective modules.
Let~$M$ be a module over a ring~$R$. Then the induced quasicoherent
module~$M^\sim$ on~$\Spec A$ is a vector bundle (\ie a finite free sheaf of
modules) if and only if~$M$ is projective and finitely generated. It is a
trivial bundle if and only if~$M$ is finite free. There is an analogous
relation in the smooth setting: The category of smooth vector bundles on a
compact manifold~$M$ is, by taking global sections, equivalent to the category
of finitely generated projective modules over~$\C^\infty(M)$.
\end{ex}

The reason why injective and projective objects are important in homological
algebra is that they \emph{have great exactness properties}. The following
lemma makes one of this properties precise. Injective and projective objects
thus form the building blocks by which other objects are \emph{resolved by} --
see the next section.

\begin{lemma}Let~$F : \A \to \B$ be an additive functor between abelian
categories. Let~$X^\bullet$ be a bounded below complex of injective objects or
a bounded above complex of projective objects. If~$X^\bullet$ is acyclic,
then~$F(X^\bullet)$ is acyclic as well~(!).\end{lemma}

Note that the statement is totally false without the injectivity or
projectivity assumption.

\begin{proof}One can show that that such a complex is homotopy equivalent to
the zero complex. Any additive functor preserves homotopy equivalences.
Therefore the image complex is too homotopy equivalent to zero and in
particular acyclic.

The already interesting special case of three-term complexes, \ie short exact
sequences, can be proved by a simpler argument: If~$0 \to A \to B \to C \to 0$
is a short exact sequence with~$A$ injective or~$C$ projective, one can
construct a retraction of~$A \to B$ respectively a section of~$B \to C$.
Therefore the sequence splits. The claim follows since additive functors
preserve biproducts.\end{proof}

\begin{itemize}
\item Check statement about vector bundles
\end{itemize}

\section{Resolutions}

\begin{itemize}
\item Definition
\item Existence
\item Morphisms
\item Uses: good replacements
\item Motto: $X = I^0 - I^1 + \ldots$
\item Construction of free resolutions
\item Koszul resolutions
\item Homological dimension
\end{itemize}

\section{Derived functors}

\begin{defn}Let~$F : \A \to \B$ be a left-exact functor between abelian
categories. Assume that~$\A$ has enough injectives. Then~$\RR F(X)$, the
\emph{total right-derived functor of~$F$} evaluated at an object~$X$ of~$\A$,
is the complex
\[ \RR F(X){:}\quad \cdots \to 0 \to F(I^0) \to F(I^1) \to \cdots, \]
where~$0 \to X \to I^\bullet$ is an injective resolution of~$X$. The
\emph{$n$-th right-derived functor of~$F$} evaluated at~$X$ is the~$n$-th
cohomology of this complex:
$R^n F(X) \defeq H^n(\RR F(X)) = H^n(F(I^\bullet))$.
\end{defn}

So, to calculate~$R^n F(X)$, we take an injective resolution of~$X$, apply~$F$
to this complex (be sure to strip the leading~``$F(X)$''), and then take
cohomology. Note that, even though the complex~$I^\bullet$ is not acyclic only
at position zero, the complex~$F(I^\bullet)$ may have cohomology in arbitrary
high degrees~(!). There is an analogous definition for the left-derived
functors of right-exact functors, using projective instead of injective
resolutions.

Since we made the arbitrary choice of an injective resolution, we have to
discuss well-definedness of course. If~$0 \to X \to J^\bullet$ is a second
injective resolution, there is a homotopy equivalence~$I^\bullet \simeq
J^\bullet$ by Lemma~\ref{lemma:inj-he}. This equivalence is preserved by the
functor~$F$, so the complexes~$F(I^\bullet)$ and~$F(J^\bullet)$ are homotopy
equivalent as well. In particular, they are quasi-isomorphic and have
isomorphic cohomology.

Summing up, the complex~$\RR F(X)$ is well-defined up to quasi-isomorphism and~$R^n F(X)$
is well-defined up to isomorphism. For reasons to be stated below, we don't
want to stress that with our definition,~$\RR F(X)$ is actually even
well-defined up to homotopy equivalence.

Suppose we fix injective resolutions for every object of~$\A$. Then the rule~$X
\mapsto R^n F(X)$ can be made into an additive functor~$R^n F : \A \to \B$,
since morphisms~$X \to Y$ between objects induce morphisms between their
associated injective resolutions (which are unique up to homotopy) and
therefore well-defined morphisms~$R^n F(X) \to R^n F(Y)$.

Different choices of resolutions will induce naturally isomorphic right-derived
functors.

\begin{ex}We have~$R^0 F(X) \cong F(X)$, naturally in~$X$. To verify this, pick
an injective resolution~$0 \to X \to I^\bullet$ and apply~$F$ to
the exact sequence~$0 \to X \to I^0 \to I^1$.
\end{ex}

It is in this sense that the~$R^n F$ are ``derived'' functors of~$F$; $R^0 F$
coincides with~$F$ and the higher derived functors bear some relation with~$F$.
But note that~$R^{n+1} F$ is \emph{not} the first derived functor of~$R^n F$
(indeed, in general the higher derived functors are neither left- nor
right-exact, so we cannot derive~$R^n F$) and that there is no Leibniz rule to
be found.

\begin{ex}Assume~$F$ to even be exact. Then the higher derived functors of~$F$
vanish, \ie~$R^n F(X) = 0$ for all objects~$X$ and~$n > 0$. To verify this, use
Lemma~\ref{lemma:inj-acyclic}.\end{ex}

This suggests the following motto: \emph{Derived functors measure the
failure of a functor to be exact.}

\begin{ex}Consider the dualization functor~$F : \Ab^\op \to \Ab,\,M \mapsto
\Hom(M,\ZZ)$. Since~$F$ is an instance of a Hom functor, it is left-exact. To
ensure that the contravariance doesn't cause unnecessary difficulties, we spell
out precisely what this means: If~$0 \to M \to N \to P \to 0$ is a short exact
sequence in~$\Ab^\op$ (this really means that we have a short exact sequence~$0
\leftarrow M \leftarrow N \leftarrow P \leftarrow 0$ in~$\Ab$), then~$0 \to F(M) \to F(N) \to
F(P)$ is exact in~$\Ab$. To compute~$\RR F(\ZZ/(2))$, we use the projective
(even free) resolution
\[ 0 \lra \ZZ \stackrel{2}{\longrightarrow} \ZZ \lra \ZZ/(2) \lra 0, \]
which corresponds to an injective resolution in~$\Ab^\op$. The total derived
functor is therefore the complex
\[ 0 \lra F(\ZZ) \stackrel{2}{\longrightarrow} F(\ZZ) \lra 0, \]
which can be simplified using~$F(\ZZ) \cong \ZZ$ to
\[ 0 \lra \ZZ \stackrel{2}{\longrightarrow} \ZZ \lra 0. \]
Thus we obtain~$R^0 F(\ZZ) \cong F(\ZZ) = 0$ and~$R^1 F(\ZZ) \cong \ZZ/(2)$.
Note that the torsion did not vanish, but instead moved to~$R^1$.
\end{ex}

Many important functors are derived functors. For instance,
\begin{itemize}
\item the right derived functors of~$\Hom(X, \placeholder) : \A^\op \to
\Ab$ are the \emph{Ext functors} $\Ext^n(X, \placeholder)$,
\item the left derived functors of~$M \otimes_A \placeholder : \Mod(A) \to \Ab$
are the \emph{Tor functors} $\Tor_n(M, \placeholder)$,
\item the right derived functors of the global sections functor~$\Gamma :
\AbSh(X) \to \Ab$ calculate \emph{sheaf cohomology}, and
\item the right derived functors of the functor~$\Mod(G) \to \Ab$ which associates to
a~$G$-module its subset of~$G$-invariant elements calculate \emph{group
cohomology}.
\end{itemize}

The raison d'être for derived functors is given by the following lemma.

\begin{lemma}Let~$F : \A \to \B$ be a left-exact functor between abelian
categories. Assume that~$\A$ has enough injectives. Then, for any short exact
sequence~$0 \to X \to Y \to Z \to 0$ in~$\A$, there is an induced long exact
sequence
\[ 0 \to F(X) \to F(Y) \to F(Z) \to R^1 F(X) \to R^1 F(Y) \to R^1 F(Z) \to R^2
F(X) \to \cdots, \]
depending functorially on the short exact sequence.\end{lemma}
\begin{proof}Use the \emph{horseshoe lemma} to obtain a short exact sequence
relating injective resolutions of~$X$, $Y$, and~$Z$. Then perform the usual
diagram chase to construct the connecting morphisms.\end{proof}

Note that the lemma shows very visibly that~$F$ is exact if and only if~$R^1 F
= 0$.

In practice, calculating the higher derived functors can be quite hard. There
are two main techniques: One can use exact sequences (and spectral sequences)
to draw some conclusions about the derived functors, or hope that there are
more amenable resolutions than resolutions by unwieldy injectives available. To
this end, the following lemma is crucial:

\begin{lemma}Let~$F : \A \to \B$ be a left-exact functor between abelian
categories. Assume that~$\A$ has enough injectives. Let~$X$ be an object
of~$\A$. Let~$0 \to X \to U^\bullet$ be a resolution of~$X$ by
\emph{$F$-acyclic objects}, \ie objects such that~$R^{\geq 1}F(U^m) = 0$.
Then~$\RR F(X)$ is quasi-isomorphic to~$F(U^\bullet)$ and in particular~$R^n
F(X) \cong H^n(F(U^\bullet))$.\footnote{Note that~$\RR F(X)$ will in general
\emph{not} be homotopy equivalent to~$F(U^\bullet)$. This is the reason why we
identify~$\RR F(X)$ up to quasi-isomorphism and not up to homotopy equivalence.}\end{lemma}

In other words, we may use resolutions by~$F$-acyclic objects instead of
injective objects to calculate the derived functors. For instance, we can use
flat resolutions to calculate~$\Tor$ and flabby resolutions to calculate sheaf
cohomology.

Derived functors can be used to fix several defects in the original functors. The
paramount defect of a non-exact functor is, of course, its failure to preserve
exactness. This is fixed by the long exact sequence. But derived
functors also fix the failure of certain identities. For instance, if~$M$ is an
arbitrary~$A$-module, in general it does not hold that
\[ M^{\vee\vee} \defeq \Hom(\Hom(M,A),A) \cong M. \]
But, denoting by~$(\placeholder)^{\dd}$ the total derived functor of the
dualization functor, the complex~$M^{\dd\dd}$ \emph{is} quasi-isomorphic to~$M$
(regarded as a complex concentrated in degree zero).\footnote{We are skating
over some details here. In particular, we have not explained how to apply~$\RR
F$ to a complex~$K^\bullet$ (instead of a single object); this is done by
finding a quasi-isomorphism~$K^\bullet \to I^\bullet$ to a complex consisting
of injectives and setting~$\RR F(K^\bullet) \defeq F(I^\bullet)$. Also one has
to impose some finiteness restrictions, for instance~$A$ should be Noetherian
and~$M$ finitely generated.} Similarly, in algebraic geometry, there is the
\emph{projection formula}
\[ \pi_*\E \otimes \F \cong \pi_*(\E \otimes \pi^*\G), \]
valid in the case that~$\G$ is a locally free sheaf.
For the derived functors, the formula holds in complete generality, assuming
only some finiteness conditions:
\[ \RR\pi_*(\E^\bullet) \otimes^\LL \F^\bullet \simeq_\text{qis}
\RR\pi_*(\E^\bullet \otimes^\LL \LL\pi^*\G^\bullet). \]


\begin{itemize}
\item somewhere: ``complexes good, cohomology bad'', example from homotopy theory
\end{itemize}

\section{Example: Ext}

\begin{itemize}
\item Lifting problems
\item Extensions (give correspondence in footnote)
\end{itemize}

\section{Example: Sheaf cohomology}

\begin{itemize}
\item Resolution of $\underline{\RR}$
\item Comparison with singular cohomology
\item Paucity of projective sheaves
\item Example: section-wise surjectivity
\item Uninteresting: $\C^\infty$-modules
\end{itemize}

\section{Outlook}

\begin{itemize}
\item Unbounded resolutions
\item Derived categories: definition, composition of derived functors,
$D^b(X)$ $f^!$
\item Model categories: $\mathrm{sAb} \simeq \mathrm{Kom}^{\leq
0}(\mathrm{Ab})$
\end{itemize}

\end{document}
